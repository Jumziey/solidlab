\documentclass[a4paper,12pt]{article}
\usepackage{amsmath}
\usepackage{amsfonts}
\usepackage[english]{babel}
\usepackage{fancyhdr}
%\usepackage[prefix]{nomencl}
%\usepackage{makeidx}
\usepackage{cite}
\usepackage{float}
\usepackage{placeins}
\usepackage{graphicx}
\usepackage{epstopdf}
%\usepackage{multicol}
%\usepackage{url}
\usepackage{listings}
\usepackage{pdfpages}
\usepackage[utf8]{inputenc}
\usepackage[hidelinks]{hyperref}
%\usepackage{cleveref}

\usepackage{caption}
%\usepackage{subcaption}
\usepackage{subfig}
\renewcommand{\contentsname}{Innehållsförteckning}
%\hypersetup{linkcolor=blue, colorlinks=true}


%\makenomenclature

\let\oldAuthor\author
\renewcommand{\author}[1]{\newcommand{\myAuthor}{#1}\oldAuthor{#1}}
\renewcommand{\deg}{\ensuremath{^{\circ}}\xspace}

\begin{document}

%---------------------------------------------------------
%Fill in:
% Redaktör
%---------------------------------------------------------

%---------------------------------------------------------
%Fill in:
%1. Redaktör(er)
%3. Organisation/Företag/Institution...
%4. Projektnamn
%5. Projektgrupp, med email eller liknande kontakt uppgift
%6. Granskare av dokument
%7. Datum som graskningen skedde
%8. Godkännare av dokumentet
%9. Datum för godkännande
%---------------------------------------------------------
\title{Project report}
\author{Tobias Nilsson }
\date{} %<-- LEAVE EMPTY! 
\newcommand{\organisation}[0] {\small Institutionen för Fysik \\ Umeå Universitet}
\newcommand{\projektnamn}{\small Upgrade: Electrical Conductivity}
\newcommand{\projektgrupp}{ Tobias Nilsson, toni0042@student.umu.se \\
							 Jesper Vesterberg, jeve0100@student.umu.se}
\newcommand{\granskare}{Jesper Vesterberg}
\newcommand{\granskatdatum}{GRANSKATTDATUM}
\newcommand{\godkannare}{Krister Wiklund}
\newcommand{\godkantdatum}{GODKÄNTDATUM}


\begin{titlepage}
\maketitle 
\thispagestyle{fancy}
\headheight 35pt 
\lhead{\organisation}
\chead{\projektnamn}
\rhead{\small\today}
\cfoot{\projektgrupp}

\begin{center}
Version 1.0
\end{center}

\vspace{80mm}
\begin{center}
  {\large Status:}\\[1.5ex]
  \begin{tabular}{|*{3}{p{40mm}|}}
    \hline
    Reviewed & \granskare & \granskatdatum \\
    \hline
    Approved & \godkannare & \godkantdatum \\
    \hline
  \end{tabular}
\end{center}

\end{titlepage}



\pagestyle{fancy}
\headheight 35pt 
\pagenumbering{roman}
\rhead{\small\today \\ }
\chead{\projektnamn}
\lhead{\organisation \\ }
\lfoot{Projektarbete inom \\teknisk fysik B, 3.0 HP \\ 5FY070}
\cfoot{\thepage}
\rfoot{\projektgrupp}
\begin{center}

\section*{\center Project identity}

\bigskip
\begin{tabular}{|p{35mm}|p{30mm}|p{20mm}|p{45mm}|}
\hline
\textbf{Name} & \textbf{Ansvar} & \textbf{Telephone} & \textbf{e-mail}\\
\hline
Tobias Nilsson & Documents &  & toni0042@student.umu.se\\
\hline
Jesper Vesterberg & Programmer &  & jeve0010@student.umu.se\\
\hline
\end{tabular}

\bigskip
\textbf{Client}: Institutionen för Fysik, Umeå universitet, \\
Linnaeus väg 24,\\
901 87\\
Umeå.

\textbf{Contact}: Krister Wiklund, krister.wiklund@umu.se.
\end{center}
\newpage

\tableofcontents
\newpage
\pagenumbering{arabic}

\subsection{Introduction}
The lab \emph{Electrical conductivity} in the courses \emph{Solid State Physics} was upgraded from an old set-up which consisted of; a cryo system, with heat controller; a series of power supplies; a black box with measurement circuits, that converted resistance to voltage; a bench multimeter and an ancient  computer system running on MS-DOS. The upgrades consisted of removing the external power supplies, the black box and exchanging the computer for a newer laptop and the user interface, \emph{Elledn}, was rewritten in Labview.

The purpose of the the laboratory is to investigate how the electrical conductivity of different materials behaves in the temperature region $10-300$ $K$. The materials are a semi-conductor, InSb, a metal, Pt, and a super conductor, $YBa_2Cu_3O_7$.

\section{The set-up}
Equipment consist of:
\begin{itemize}
\item Cryogenic system:
\begin{itemize}
\item Cooling chamber
\item Coarse vacuum pump
\item turbo molecular pump
\item Cryogenic system(compressor w. cooling system): HC-2, APD Cryogenics Inc.
\end{itemize}
\item Controllers to cryogenic system: 
\begin{itemize}
\item Turbotronic NT10
\item Thermovac TM20
\item Scientific Instruments Inc. 9620-1 Silicon Diode
\end{itemize} 
\item Bench Multimeter: Keithley 2001 Multimeter
\item Computer with Labview.
\end{itemize}

The materials are placed in the cooling chamber and the two vacuum pumps reduces the pressure to below $10^{-5}$ Pascal. This reduces the amount of heat that needs to be removed to change the chambers temperature, remember the ideal gas law $PV=NkT$. The actual cooling process is a closed helium-gas cycle following the Gifford-McMahon principle.

The samples are continuously cooled by the equipment to $\approx 10$ $K$ if no additional heat is added. An external heater is used to alter the temperature by adding heat to the system. This heater is controlled by the \emph{Scientific Instruments Inc. 9620-1 Silicon Diode}. Which in turn can be controlled by a computer through an GPIB-interface and its front panel. 

The controller determines the heating with the aid of en equation consisting of an integral and a differential term. To improve the performance and responsiveness of the cryogenic equipment in various ranges, these terms can be changed by the user. For more information see the \emph{Scientific Instruments Inc. 9620-1 Silicon Diode manual}.

To determine the resistivity of the samples, each sample is connected to a 4 sense wire system. These sense wires are connected to a Keithley 2001 Multimeter through a switching card at the back of the multimeter. The multimeter is also equipped with a GPIB-interface, enabling computer controlled measurements. 

The connections from the samples and an additional temperature diode, goes first a through a round 19 pin contact to an 25 pin d-sub connector. See table \ref{tab:PinNumbers} for pin layout on the cables.

\begin{table}
	\label{tab:PinNumbers}
  \caption{\emph{Pin layout for connection cables between cooling chamber and multimeter. Ports 19-24 on the 25-pin Dsub connector are  unused.}} 
  
  \begin{tabular}{c|c|c c c}
  25-pin D-sub & round 19-pin connector & Function & & \\
  \hline
  1  & A & Superconductor    & Current & + \\
  2  & B & ''                & ''      & - \\
  3  & C & ''                & Voltage & + \\
  4  & D & ''                & ''      & - \\
  5  & E & Conductor, Pt-100            & Current & + \\
  6  & F & ''                & ''      & - \\
  7  & G & ''                & Voltage & + \\
  8  & H & ''                & ''      & - \\
  9  & J & Semi-conductor, InSb-plate        & Current & + \\
  10 & K & ''                & ''      & - \\
  11 & L & ''                & Voltage & + \\
  12 & M & ''                & ''      & - \\
  13 & N & Temperature diode & Current & + \\
  14 & P & ''                & ''      & - \\
  15 & R & ''                & Voltage & + \\
  16 & S & ''                & ''      & - \\   
  17 & T & free              &         &  \\
  18 & U & free              &         &  \\
  25 & V & Shield            &         &  \\ 
  \end{tabular}
\end{table} 


\begin{table}
	\center
	\label{tab:ScanCard}
	\caption{\emph{Pin layout for the Scan card, inside the Keithley Multimeter.}} 
	\begin{tabular}{l|l|l}
		\#  & +(Red) & -(Black) \\
		1  & Superconductor, Potential +	& Superconductor, Potential - \\
		2  & Semi-conductor, Potential +		& Semi-conductor, Potential - \\
		3  & Conductor, Potential +			& Conductor, Potential - \\
		4  & 	N/A	& N/A \\
		5  & 	N/A	& N/A \\
		6  & Superconductor, Current +		& Superconductor, Current - \\
		7  & Semi-conductor, Current +		& Semi-conductor, Current - \\
		8  & Conductor, Current +			& Conductor, Current - \\
		9  & 	N/A	& N/A \\
		10 & 	N/A	& N/A \\
	\end{tabular}
\end{table}

The samples are connected to the Keithley Multimeter through a custom scanning card. This card enables the use of the built-in switching capabilities and it has two rows of pins, a red (positive) and a black (negative). When using this home made card for 4-wire measurements it will can be important to know that the first five pins (pins 1-5) are used for the potential wires and that the  last five (pins 6-10) are used for the current wires.

The first potential pins are grouped with the first current pins, e.g. pins 1 and 6 are connected two sample 1 and pins 2 and 7 are connected to sample 2 and so on. See Table \ref{tab:ScanCard} for details about how the samples currently are connected to the scanning card.
 

\section{Software}

This section describes shortly how the software work, for a guide on how to operate the software see the laboratory instructions for electrical conductivity. 

The software, ResSolidLabV1, is written in Labview and it   has a reduced control of the measurement system. For simplicity it only change the temperature and initiate measurements. All other changes in the measurement system, has to be done outside the user interface. Such as changing the different coefficients in the equation that controls the heating element, which are done in the heating element's controller. 

However the bench-multimeter's settings are made automatically by ResSolidLabV1, these settings are about how the measurements are performed and the change these settings, the parts of the program needs to be rewritten.

JESPER! FYLL I HÄR VILKA SAKER SOM ÄR "HÅRDKODADE" I PROGRAMMET OCH HUR MAN ÄNDRAR PÅ DEM!!!!!!!!!!!!!!

The  bench-multimeter can not truly measure zero resistance even with the 4-wire setup and the measured values are often negative. The solution implemented in the software is to chose the maximum value between the measured resistance and zero, this introduces a potential problem. Switching place of either the current or resistance cables will change the sign of the measure resistance.


\section{Hardware}

\subsection{Keithley 2001 Multimeter}

VILKA KOMMANDON FINNS DET, VILKA PORTAR PÅ DET HEMMA GJORDA KORTET HÖR IHOP.

\subsubsection{GPIB-Commands: Multimeter}

\subsection{Heater control: \\ Scientific Instruments Inc. 9620-1 Silicon Diode}
Scientific Instruments Inc. 9620-1 Silicon Diode is the controller that sets the heat, as previous stated it governed by an equation containing a differential and an integral term. These terms can be set by the user using the controller's front panel but also by sending commands over the GPIB-interface.

Though the current settings seems to work just fine for the present usage so there are no reason the temper with these settings.

\subsubsection{GPIB-Commands: Heater}




%\begin{appendix}


%Here you can put additional data of importance Labview or Matlab codes,
%additional equations and derivations etc.
%Do not add long tables of raw data. 
%
%For programming code like Matlab you need to use a font where 
%all letters are equally wide. Use the verbatim environment.
%
%\begin{verbatim}
%	Paste code here! 
%\end{verbatim}

%\section{Appendix section 2} 

%\end{appendix}

\end{document}